\documentclass{article}
\usepackage{amsmath}

\newcommand{\reducesto}{{\le}_m^p}
\newcommand{\argmax}{\text{argmax}}
\newcommand{\TWOSAT}{{\lang{2SAT}}}
\newcommand{\sC}{\mathcal{C}}
% \usepackage{fullpage}
\usepackage{graphicx}
\usepackage{enumerate}
\usepackage{fancyhdr}

\pagestyle{fancy}

%% L/C/R denote left/center/right header (or footer) elements
%% E/O denote even/odd pages

%% \leftmark, \rightmark are chapter/section headings generated by the 
%% book document class

\fancyhead[R]{\slshape Ritika Goel, Arnold Noronha}
\fancyhead[L]{\slshape CIS-540, Homework 4} 
\fancyfoot[L]{}
\fancyfoot[C]{}
\fancyfoot[R]{\slshape Page \thepage  \text{ of  4}}

%\pagestyle{fancy}
\begin{document}

\title{CIS540, Homework 4}\author{Ritika Goel, Arnold Noronha
}

\maketitle

\section{}

\subsection*{(a)}

The state is:

$$ x(t) = \left[ \begin{array}{cc}
\varphi(t) \\
\dot{\varphi}(t)
\end{array}
\right]$$

The transition functions are, 
\begin{eqnarray}
 f_1 (x, u) &=& x_2 \\
 f_2 (x, u) &=& \frac{g}{l} \sin(x_1) - u
\end{eqnarray} 

\subsection*{(b)}

Since the system is in equilibrium, we have two conditions, 
\begin{eqnarray}
f_1 (x, 0) &=& 0 \\
f_2 (x, 0) &=& 0 
\end{eqnarray}

The first condition gives $x_2 = \dot{\varphi}(t) = 0$, and the second
condition gives, $g\sin x_1 /l = 0$ or $x_1 = \varphi(t) = 0, n\pi$.

For the case, $\langle 0, 0\rangle$, if there is a small deviation in
$\varphi$ and velocity is zero, then the pendulum can never reach the
position $\langle 0,0\rangle$, because this equilibrium position has
a higher potential energy than the initial position.

For $\langle \pi, 0\rangle$, the energy of the system is 0, however, any
epsilon deviation has some potential energy in it, so if it ever
reaches $\varphi = \pi$, the velocity cannot be zero at the same time.

\subsection*{(c)}


The linearized differential equation:
$$-ml^2 \ddot {\varphi} (t) + mgl \varphi(t) = u(t).$$

This corresonds to the following update function:
$$ f_2 (x,u) = \frac{gx_1}{l} - \frac{u}{ml^2}.$$


In other words, we can write the transition as:
$$ \dot{x}(t) = \left[ \begin{array}{cc} 
0 & 1\\
g/l & 0
\end{array} \right] x(t) + \left[\begin{array}{cc}
 0 \\
-1/ml^2
\end{array}\right] u(t)
.$$


\subsection*{(d)}

If we set $u(t) = \alpha \varphi(t) + \beta \dot{\varphi}(t)$, then
the transition  becomes:
$$ \dot{x}(t) = \left[ \begin{array}{cc} 
0 & 1\\
g/l - \alpha/ml^2 & -\beta/ml^2
\end{array} \right] x(t).$$

The eigenvalues can be of this equation can be computed as the
solution of 
\begin{eqnarray}
 \det\left| \begin{array}{cc} 
0 - \lambda & 1\\
g/l - \alpha/ml^2 & -\beta/ml^2-\lambda 
\end{array} \right| &=& 0, \\
 \lambda^2 + \frac{\beta}{ml^2}\lambda   +
 \left(\frac{\alpha}{ml^2} - \frac{g}{l}\right) &=& 0,
\end{eqnarray}

From which we get, 
$$\lambda = \frac{1}{2} \left(\frac{-\beta}{ml^2} \pm \sqrt {\frac{\beta^2}{m^2l^4}
 - 4 \left(\frac{\alpha}{ml^2} - \frac{g}{l}\right)}\right).$$


For the real parts of both solutions to be negative, clearly $\beta >
0$. Now we have two cases:
\begin{enumerate}
\item if solutions are real, i.e., 
$  \frac{\beta^2}{m^2l^4}
 - 4 \left(\frac{\alpha}{ml^2} - \frac{g}{l}\right) \ge 0,$ then we
   can we just need to test that, $\frac{\beta^2}{m^2l^4} >  \frac{\beta^2}{m^2l^4}
 - 4 \left(\frac{\alpha}{ml^2} - \frac{g}{l}\right)$. From these two
 inequalities we get
    $$  mgl <\alpha \le mgl + \frac{\beta^2}{4ml^2} .$$

\item if the solutions are complex:  the real part is always
  negative if $\beta > 0$. But this case only if $\alpha  > mgl +
  \frac{\beta^2}{4ml^2}$.
\end{enumerate}

Thus the system is stable for all $\beta > 0$ and $\alpha > mgl$.

\end{document}
